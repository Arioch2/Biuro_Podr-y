\chapter{Wprowadzenie}

Niniejszy dokument stanowi dokumentację techniczną projektu desktopowej aplikacji wspomagającej pracę biura podróży. System pozwala użytkownikom na przeglądanie ofert wycieczek, dokonywanie rezerwacji oraz zarządzanie danymi w przyjaznym, graficznym interfejsie. Aplikacja została wykonana w języku \texttt{Java} z wykorzystaniem technologii \texttt{Swing}, \texttt{JDBC} oraz relacyjnej bazy danych \texttt{PostgreSQL}.

\section{Cel i zakres projektu}

Celem projektu było stworzenie funkcjonalnej aplikacji, która umożliwia:
\begin{itemize}
    \item przeglądanie i filtrowanie dostępnych ofert wycieczek według daty i ceny,
    \item rejestrację oraz logowanie użytkowników wraz z walidacją danych,
    \item dokonywanie i przeglądanie rezerwacji,
    \item administrowanie danymi użytkowników i ofert przez dedykowany panel administracyjny.
\end{itemize}

Projekt obejmuje również przygotowanie szczegółowej dokumentacji wykonawczej w systemie \LaTeX.

\section{Wymagania systemowe}

Minimalne środowisko niezbędne do uruchomienia aplikacji obejmuje:
\begin{itemize}
    \item system operacyjny: Windows 10 lub nowszy,
    \item zainstalowane środowisko \texttt{Java Development Kit 17} lub wyższe,
    \item relacyjną bazę danych \texttt{PostgreSQL 15} wraz z narzędziem \texttt{pgAdmin 4},
    \item zainstalowany sterownik \texttt{PostgreSQL JDBC},
    \item minimum 4 GB pamięci operacyjnej RAM.
\end{itemize}

\section{Zastosowane technologie}

W projekcie wykorzystano następujące rozwiązania technologiczne:
\begin{itemize}
    \item \textbf{Java SE 17} — do implementacji logiki aplikacji,
    \item \textbf{Swing} — do tworzenia graficznego interfejsu użytkownika,
    \item \textbf{JDBC} — jako warstwę połączeniową z bazą danych,
    \item \textbf{PostgreSQL 15} — jako system zarządzania bazą danych,
    \item \textbf{\LaTeX} — do opracowania dokumentacji technicznej.
\end{itemize}
