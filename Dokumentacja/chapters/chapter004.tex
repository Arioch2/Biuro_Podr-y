\chapter{Harmonogram, testy i podsumowanie}

\section{Harmonogram realizacji projektu}

Projekt został zrealizowany w okresie od maja do czerwca 2025 roku. Całość prac została podzielona na trzy główne etapy:

\begin{itemize}
    \item \textbf{maj 2025} — zaprojektowanie struktury bazy danych oraz graficznego interfejsu użytkownika,
    \item \textbf{maj–czerwiec 2025} — implementacja logiki aplikacji oraz integracja z bazą danych PostgreSQL,
    \item \textbf{czerwiec 2025} — przeprowadzenie testów, wprowadzenie poprawek oraz przygotowanie dokumentacji technicznej.
\end{itemize}

Wizualny przebieg realizacji przedstawiono na diagramie Gantta (rys.~\ref{fig:gantt}).

\begin{figure}[H]
    \centering
    \includegraphics[width=0.95\textwidth]{img/gantt.png}
    \caption{Diagram Gantta przedstawiający harmonogram projektu}
    \label{fig:gantt}
\end{figure}

\section{Repozytorium i wersjonowanie}

Do kontroli wersji zastosowano system \texttt{Git}, natomiast kod źródłowy przechowywano w publicznym repozytorium serwisu GitHub:

\begin{itemize}
    \item \textbf{Repozytorium:} \url{https://github.com/Arioch2/Biuro_Podr-y}
    \item \textbf{Główna gałąź:} \texttt{main}
\end{itemize}

Rozwiązanie to umożliwia łatwą kontrolę nad postępem prac oraz historią zmian.

\section{Testowanie aplikacji}

Testy zostały przeprowadzone zarówno manualnie, jak i poprzez proste testy jednostkowe dla wybranych fragmentów logiki biznesowej. Sprawdzono poprawność działania formularzy, walidacji danych oraz połączenia z bazą danych.

\subsection{Walidacja danych wejściowych}

Walidacja danych odbywa się głównie po stronie klienta z wykorzystaniem klasy \texttt{Validator.java}. Przykładowy fragment tej klasy przedstawiono na listingu \ref{lst:validator}.

\lstinputlisting[style=javaStyle, caption={Fragment klasy \texttt{Validator}}, label={lst:validator}]{src/Validator.java}

\subsection{Połączenie z bazą danych}

Kluczową rolę w obsłudze połączenia z bazą danych pełni klasa \texttt{DatabaseConnector}. Listing \ref{lst:dbconnector} prezentuje fragment tej klasy, odpowiedzialny za inicjalizację połączenia JDBC.

\lstinputlisting[style=javaStyle, caption={Fragment klasy \texttt{DatabaseConnector}}, label={lst:dbconnector}]{src/DatabaseConnector.java}

\section{Podsumowanie i wnioski}

Zaprojektowana i zaimplementowana aplikacja spełnia wszystkie zakładane cele funkcjonalne. Użytkownik końcowy ma możliwość rejestracji, logowania, przeglądania i rezerwacji ofert. Z kolei administrator systemu otrzymuje dostęp do rozszerzonych funkcji zarządzania.

Potencjalne kierunki dalszego rozwoju systemu obejmują:

\begin{itemize}
    \item implementację mechanizmu wysyłania potwierdzeń e-mailowych po dokonaniu rezerwacji,
    \item rozszerzenie systemu o możliwość eksportowania danych do plików PDF,
    \item integrację z zewnętrznymi usługami, takimi jak mapy lub prognozy pogody,
    \item dostosowanie interfejsu użytkownika do urządzeń mobilnych.
\end{itemize}
