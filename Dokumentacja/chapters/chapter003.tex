\chapter{Struktura systemu i baza danych}

\section{Architektura aplikacji}

Aplikacja została opracowana zgodnie z wzorcem projektowym \texttt{MVC} (Model–View–Controller), co zapewnia przejrzysty podział odpowiedzialności pomiędzy poszczególne komponenty systemu:

\begin{itemize}
    \item \textbf{Model} – klasy reprezentujące struktury danych (np. użytkownik, oferta, rezerwacja),
    \item \textbf{View} – komponenty interfejsu graficznego zbudowane w oparciu o bibliotekę \texttt{Swing},
    \item \textbf{Controller} – klasy pośredniczące pomiędzy logiką aplikacji, bazą danych a interfejsem graficznym.
\end{itemize}

\section{Baza danych}

System przechowuje dane w relacyjnej bazie \textbf{PostgreSQL 15}. Projekt bazy został przygotowany w środowisku \texttt{pgAdmin 4}, a komunikacja z aplikacją odbywa się za pośrednictwem sterownika \texttt{PostgreSQL JDBC} w wersji 42.7.5.

\subsection{Tabela \texttt{users}}

Tabela przechowuje podstawowe informacje rejestracyjne użytkowników:

\begin{table}[H]
\centering
\begin{tabular}{|l|l|l|}
\hline
\textbf{Kolumna} & \textbf{Typ} & \textbf{Opis} \\ \hline
id & SERIAL PRIMARY KEY & ID użytkownika \\
login & VARCHAR(30) UNIQUE & Nazwa logowania \\
email & VARCHAR(50) UNIQUE & Adres e-mail \\
password & VARCHAR(100) & Zaszyfrowane hasło \\
role & VARCHAR(10) & Rola systemowa (\texttt{user} lub \texttt{admin}) \\ \hline
\end{tabular}
\caption{Struktura tabeli \texttt{users}}
\end{table}

\subsection{Tabela \texttt{offers}}

Zawiera dane związane z ofertami turystycznymi:

\begin{table}[H]
\centering
\begin{tabular}{|l|l|l|}
\hline
\textbf{Kolumna} & \textbf{Typ} & \textbf{Opis} \\ \hline
id & SERIAL PRIMARY KEY & ID oferty \\
title & VARCHAR(100) & Nazwa wycieczki \\
description & TEXT & Szczegóły oferty \\
price & DECIMAL(10,2) & Cena brutto \\
start\_date & DATE & Data rozpoczęcia \\
end\_date & DATE & Data zakończenia \\ \hline
\end{tabular}
\caption{Struktura tabeli \texttt{offers}}
\end{table}

\subsection{Tabela \texttt{reservations}}

Zawiera dane dotyczące dokonanych rezerwacji:

\begin{table}[H]
\centering
\begin{tabular}{|l|l|l|}
\hline
\textbf{Kolumna} & \textbf{Typ} & \textbf{Opis} \\ \hline
id & SERIAL PRIMARY KEY & ID rezerwacji \\
user\_id & INTEGER & Klucz obcy do tabeli \texttt{users} \\
offer\_id & INTEGER & Klucz obcy do tabeli \texttt{offers} \\
reservation\_date & TIMESTAMP & Data złożenia rezerwacji \\ \hline
\end{tabular}
\caption{Struktura tabeli \texttt{reservations}}
\end{table}

\section{Diagram relacji encji}

Schemat powiązań między tabelami przedstawiono na rysunku \ref{fig:erd_pgadmin}.

\begin{figure}[H]
    \centering
    \includegraphics[width=0.9\textwidth]{img/erd_pgadmin.png}
    \caption{Diagram ERD bazy danych PostgreSQL (wygenerowany w pgAdmin 4)}
    \label{fig:erd_pgadmin}
\end{figure}

\section{Organizacja pakietów i klas aplikacji}

Struktura katalogowa projektu została podzielona zgodnie z zasadami separacji odpowiedzialności:

\begin{itemize}
    \item \texttt{model} – klasy reprezentujące dane: \texttt{User.java}, \texttt{Offer.java}, \texttt{Reservation.java},
    \item \texttt{dao} – warstwa dostępu do danych: \texttt{UserDAO.java}, \texttt{OfferDAO.java},
    \item \texttt{gui} – komponenty interfejsu użytkownika: \texttt{LoginPanel.java}, \texttt{RegisterPanel.java}, \texttt{OffertsPanel.java},
    \item \texttt{util} – klasy pomocnicze: \texttt{DbConnector.java}, \texttt{Validator.java}.
\end{itemize}

Uproszczony diagram klas zaprezentowano na rysunku \ref{fig:class_diagram}.

\begin{figure}[H]
    \centering
    \includegraphics[width=0.85\textwidth]{img/class_diagram.png}
    \caption{Diagram klas aplikacji}
    \label{fig:class_diagram}
\end{figure}
