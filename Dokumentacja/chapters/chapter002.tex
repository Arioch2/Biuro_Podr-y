\chapter{Interfejs użytkownika i funkcjonalności}

\section{Logowanie i rejestracja użytkownika}

Po uruchomieniu aplikacji wyświetlany jest ekran logowania (rys. \ref{fig:login}). Użytkownik nieposiadający konta może przejść do rejestracji. Formularze logowania i rejestracji zawierają mechanizmy walidacyjne, weryfikujące poprawność adresu e-mail, długość oraz powtórzenie hasła, a także unikalność loginu na poziomie bazy danych.

\begin{figure}[H]
    \centering
    \includegraphics[width=0.7\textwidth]{img/login_form.png}
    \caption{Ekran logowania}
    \label{fig:login}
\end{figure}

\begin{figure}[H]
    \centering
    \includegraphics[width=0.7\textwidth]{img/register_form.png}
    \caption{Formularz rejestracyjny}
    \label{fig:register}
\end{figure}

\section{Panel użytkownika i oferty turystyczne}

Po zalogowaniu użytkownik trafia do głównego panelu (rys. \ref{fig:userpanel}), w którym dostępne są wszystkie możliwe akcje — m.in. podgląd rezerwacji, przeglądanie ofert wycieczek i ich filtrowanie.

\begin{figure}[H]
    \centering
    \includegraphics[width=0.9\textwidth]{img/UsersMenuPanel.png}
    \caption{Panel użytkownika}
    \label{fig:userpanel}
\end{figure}

Wybór opcji „Zobacz oferty” otwiera listę wycieczek (rys. \ref{fig:offers}), które można przefiltrować według daty oraz zakresu cenowego. Domyślnie prezentowane są wszystkie aktywne oferty.

\begin{figure}[H]
    \centering
    \includegraphics[width=0.9\textwidth]{img/offer_list_panel.png}
    \caption{Lista ofert z możliwością filtrowania}
    \label{fig:offers}
\end{figure}

\subsection{Filtrowanie według daty i ceny}

Na rysunku \ref{lst:filterOffers} przedstawiono fragment klasy \texttt{OffertsPanel.java}, realizującej filtrowanie listy ofert zgodnie z kryteriami cenowymi i daty:

\lstinputlisting[style=javaStyle, caption={Metoda filtrowania ofert po dacie i cenie}, label={lst:filterOffers}]{src/OffertsPanel.java}

\section{Panel administratora}

Użytkownik o uprawnieniach administratora po zalogowaniu uzyskuje dostęp do rozszerzonego menu systemu, umożliwiającego zarządzanie kontami użytkowników, ofertami oraz przeglądanie rezerwacji (rys. \ref{fig:adminpanel}).

\begin{figure}[H]
    \centering
    \includegraphics[width=0.9\textwidth]{img/admin_panel.png}
    \caption{Panel administratora z listą użytkowników}
    \label{fig:adminpanel}
\end{figure}

\section{Walidacja danych}

System przeprowadza walidację danych wejściowych zarówno po stronie klienta (Java Swing), jak i po stronie bazy danych. Przykładowe mechanizmy walidacyjne obejmują:

\begin{itemize}
    \item sprawdzenie poprawności adresu e-mail,
    \item weryfikację długości oraz zgodności hasła,
    \item unikalność loginu użytkownika (na poziomie bazy),
    \item poprawność danych liczbowych (np. zakres cenowy oferty).
\end{itemize}
