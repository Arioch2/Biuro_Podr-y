\documentclass{urdpl}     % praca w języku polskim

% Lista wszystkich języków stanowiących języki pozycji bibliograficznych użytych w pracy.
% (Zgodnie z zasadami tworzenia bibliografii każda pozycja powinna zostać utworzona zgodnie z zasadami języka, w którym dana publikacja została napisana.)
\usepackage[english,polish]{babel}

% Użyj polskiego łamania wyrazów (zamiast domyślnego angielskiego).
\usepackage{polski}

\usepackage[utf8]{inputenc}

% dodatkowe pakiety

\usepackage{mathtools}
\usepackage{amsfonts}
\usepackage{amsmath}
\usepackage{amsthm}
\usepackage[hidelinks]{hyperref}
\usepackage{float}
\usepackage{listings}
\usepackage{graphicx}
\usepackage{subcaption}
\usepackage{booktabs} % Dla \toprule, \midrule, \bottomrule
\usepackage{multirow} 
\usepackage{tabularx} 
\usepackage{amssymb} 
\usepackage{listings}
\usepackage{xcolor}
\usepackage{array}
\usepackage{makecell}
\usepackage[flushleft]{threeparttable}
\usepackage[normalem]{ulem}
\usepackage{lineno}
% ---------------------------------------------

% --- < bibliografia > ---

\usepackage{csquotes}

% ------------------------
% --- < listingi > ---

% Użyj czcionki kroju Courier.
\usepackage{courier}

\usepackage{listings}
\lstloadlanguages{TeX}
\renewcommand{\lstlistlistingname}{Spis listingów}
\renewcommand{\lstlistingname}{Listing}


\lstset{
	literate={ą}{{\k{a}}}1
           {ć}{{\'c}}1
           {ę}{{\k{e}}}1
           {ó}{{\'o}}1
           {ń}{{\'n}}1
           {ł}{{\l{}}}1
           {ś}{{\'s}}1
           {ź}{{\'z}}1
           {ż}{{\.z}}1
           {Ą}{{\k{A}}}1
           {Ć}{{\'C}}1
           {Ę}{{\k{E}}}1
           {Ó}{{\'O}}1
           {Ń}{{\'N}}1
           {Ł}{{\L{}}}1
           {Ś}{{\'S}}1
           {Ź}{{\'Z}}1
           {Ż}{{\.Z}}1,
	basicstyle=\footnotesize\ttfamily,
}

% defninicja stylu python
\lstdefinestyle{stylePython}{
    language=Python,
    commentstyle=\color{green},          % Kolor komentarzy
    keywordstyle=\color{blue},           % Kolor słów kluczowych
    numberstyle=\tiny\color{gray},       % Kolor i styl numerów linii
    stringstyle=\color{red},             % Kolor ciągów znaków
    basicstyle=\ttfamily\footnotesize,   % Podstawowy styl kodu
    breakatwhitespace=false,             % Automatyczne dzielenie wierszy
    breaklines=true,                     % Dzielenie długich linii
    keepspaces=true,                     % Zachowanie spacji
    numbers=left,                        % Numery linii po lewej
    numbersep=5pt,                       % Odstęp numerów od kodu
    showspaces=false,                    % Nie pokazuj spacji
    showstringspaces=false,              % Nie pokazuj spacji w ciągach znaków
    showtabs=false,                      % Nie pokazuj tabulacji
    tabsize=2                            % Rozmiar tabulacji
}

% defnicja stylu JAVA
\lstdefinestyle{javaStyle}{
    language=Java,
    basicstyle=\ttfamily\footnotesize,
    keywordstyle=\color{blue},
    commentstyle=\color{green!50!black}\itshape,
    stringstyle=\color{green},
    numberstyle=\tiny\color{gray},
    numbers=left,
    numbersep=5pt,                       % Odstęp numerów od kodu
    stepnumber=1,
    showspaces=false,                    % Nie pokazuj spacji
    tabsize=2,
    showstringspaces=false,
    breaklines=true,
    breakatwhitespace=false,             % Automatyczne dzielenie wierszy
    showtabs=false,                      % Nie pokazuj tabulacji
    keepspaces=true                    % Zachowanie spacji
}


\definecolor{stringcolor}{RGB}{163,21,21}    % pomarańczowy - stringi
\definecolor{typecolor}{RGB}{43, 145, 176}     % ciemny fiolet - klasy, typy

\lstdefinestyle{csStyle}{
    language=[Sharp]C, % dla C#; można zmienić na Java
    basicstyle=\ttfamily\footnotesize,
    keywordstyle=\color{blue},
    stringstyle=\color{stringcolor},
    commentstyle=\color{green!50!black}\itshape,
    morekeywords={class, public, private, protected, static, void, string, int, new}, % dodatkowe słowa kluczowe
    emphstyle=\color{typecolor}\bfseries, % klasy na fioletowo
    numbers=left,
    numbersep=5pt,                       % Odstęp numerów od kodu
    numberstyle=\tiny\color{gray},
    stepnumber=1,
    breaklines=true,
    showspaces=false,                    % Nie pokazuj spacji
    tabsize=2,
    showstringspaces=false,
    breakatwhitespace=false,             % Automatyczne dzielenie wierszy
    showtabs=false,                      % Nie pokazuj tabulacji
    keepspaces=true                    % Zachowanie spacji  
}

\definecolor{lightgray}{rgb}{0.9,0.9,0.9}
    % \definecolor{blue}{rgb}{0,0,1}
    \definecolor{green}{rgb}{0,0.6,0}
    % \definecolor{red}{rgb}{0.6,0,0}
    \definecolor{gray}{rgb}{0.5,0.5,0.5}

% % ------------------------
\AtBeginDocument{
	\renewcommand{\tablename}{Tabela}
	\renewcommand{\figurename}{Rys.}   
    \newcommand{\listingname}{Listing}
}


% ------------------------
% --- < tabele > ---

% defines the X column to use m (\parbox[c]) instead of p (`parbox[t]`)
\newcolumntype{C}[1]{>{\hsize=#1\hsize\centering\arraybackslash}X}

%---------------------------------------------------------------------------

\author{Wojciech Knapik}
\shortauthor{W. Knapik}
\noAlbum{134923}

\titlePL{System zarządzania Biurem Podróży napisany w języku Java}
\titleEN{Travel Agency Management System written in Java}

\shorttitlePL{System zarządzania Biurem Podróży} % skrócona wersja tytułu jeśli jest bardzo długi
\shorttitleEN{Travel Agency Management System}

\thesistype{Praca projektowa}


\thesisDone{Praca wykonana pod kierunkiem}
\supervisor{mgr Ewa Żesławska}
%\supervisor{Ewa Żesławska PhD}

\degreeprogramme{Informatyka}
%\degreeprogramme{Computer Science}

\date{2025}

\department{Instytut Informatyki}
%\department{Institute of Computer Science}

\faculty{Wydział Nauk Ścisłych i Technicznych}
%\faculty{Faculty of Science and Technology}



\setlength{\cftsecnumwidth}{10mm}

%---------------------------------------------------------------------------
\setcounter{secnumdepth}{4}
\brokenpenalty=10000\relax

% --------------------------------------------------------------------------
% główna część pracy
% --------------------------------------------------------------------------

\begin{document}

\titlepages

% Ponowne zdefiniowanie stylu `plain`, aby usunąć numer strony z pierwszej strony spisu treści i poszczególnych rozdziałów.
\fancypagestyle{plain}
{
    % Usuń nagłówek i stopkę
    \fancyhf{}
    % Usuń linie.
    \renewcommand{\headrulewidth}{0pt}
    \renewcommand{\footrulewidth}{0pt}
}

\setcounter{tocdepth}{2}
\tableofcontents
\clearpage


% dodanie poszczególnych rozdziałów 

\chapter{Wprowadzenie}

Niniejszy dokument stanowi dokumentację techniczną projektu desktopowej aplikacji wspomagającej pracę biura podróży. System pozwala użytkownikom na przeglądanie ofert wycieczek, dokonywanie rezerwacji oraz zarządzanie danymi w przyjaznym, graficznym interfejsie. Aplikacja została wykonana w języku \texttt{Java} z wykorzystaniem technologii \texttt{Swing}, \texttt{JDBC} oraz relacyjnej bazy danych \texttt{PostgreSQL}.

\section{Cel i zakres projektu}

Celem projektu było stworzenie funkcjonalnej aplikacji, która umożliwia:
\begin{itemize}
    \item przeglądanie i filtrowanie dostępnych ofert wycieczek według daty i ceny,
    \item rejestrację oraz logowanie użytkowników wraz z walidacją danych,
    \item dokonywanie i przeglądanie rezerwacji,
    \item administrowanie danymi użytkowników i ofert przez dedykowany panel administracyjny.
\end{itemize}

Projekt obejmuje również przygotowanie szczegółowej dokumentacji wykonawczej w systemie \LaTeX.

\section{Wymagania systemowe}

Minimalne środowisko niezbędne do uruchomienia aplikacji obejmuje:
\begin{itemize}
    \item system operacyjny: Windows 10 lub nowszy,
    \item zainstalowane środowisko \texttt{Java Development Kit 17} lub wyższe,
    \item relacyjną bazę danych \texttt{PostgreSQL 15} wraz z narzędziem \texttt{pgAdmin 4},
    \item zainstalowany sterownik \texttt{PostgreSQL JDBC},
    \item minimum 4 GB pamięci operacyjnej RAM.
\end{itemize}

\section{Zastosowane technologie}

W projekcie wykorzystano następujące rozwiązania technologiczne:
\begin{itemize}
    \item \textbf{Java SE 17} — do implementacji logiki aplikacji,
    \item \textbf{Swing} — do tworzenia graficznego interfejsu użytkownika,
    \item \textbf{JDBC} — jako warstwę połączeniową z bazą danych,
    \item \textbf{PostgreSQL 15} — jako system zarządzania bazą danych,
    \item \textbf{\LaTeX} — do opracowania dokumentacji technicznej.
\end{itemize}

\chapter{Interfejs użytkownika i funkcjonalności}

\section{Logowanie i rejestracja użytkownika}

Po uruchomieniu aplikacji wyświetlany jest ekran logowania (rys. \ref{fig:login}). Użytkownik nieposiadający konta może przejść do rejestracji. Formularze logowania i rejestracji zawierają mechanizmy walidacyjne, weryfikujące poprawność adresu e-mail, długość oraz powtórzenie hasła, a także unikalność loginu na poziomie bazy danych.

\begin{figure}[H]
    \centering
    \includegraphics[width=0.7\textwidth]{img/login_form.png}
    \caption{Ekran logowania}
    \label{fig:login}
\end{figure}

\begin{figure}[H]
    \centering
    \includegraphics[width=0.7\textwidth]{img/register_form.png}
    \caption{Formularz rejestracyjny}
    \label{fig:register}
\end{figure}

\section{Panel użytkownika i oferty turystyczne}

Po zalogowaniu użytkownik trafia do głównego panelu (rys. \ref{fig:userpanel}), w którym dostępne są wszystkie możliwe akcje — m.in. podgląd rezerwacji, przeglądanie ofert wycieczek i ich filtrowanie.

\begin{figure}[H]
    \centering
    \includegraphics[width=0.9\textwidth]{img/UsersMenuPanel.png}
    \caption{Panel użytkownika}
    \label{fig:userpanel}
\end{figure}

Wybór opcji „Zobacz oferty” otwiera listę wycieczek (rys. \ref{fig:offers}), które można przefiltrować według daty oraz zakresu cenowego. Domyślnie prezentowane są wszystkie aktywne oferty.

\begin{figure}[H]
    \centering
    \includegraphics[width=0.9\textwidth]{img/offer_list_panel.png}
    \caption{Lista ofert z możliwością filtrowania}
    \label{fig:offers}
\end{figure}

\subsection{Filtrowanie według daty i ceny}

Na rysunku \ref{lst:filterOffers} przedstawiono fragment klasy \texttt{OffertsPanel.java}, realizującej filtrowanie listy ofert zgodnie z kryteriami cenowymi i daty:

\lstinputlisting[style=javaStyle, caption={Metoda filtrowania ofert po dacie i cenie}, label={lst:filterOffers}]{src/OffertsPanel.java}

\section{Panel administratora}

Użytkownik o uprawnieniach administratora po zalogowaniu uzyskuje dostęp do rozszerzonego menu systemu, umożliwiającego zarządzanie kontami użytkowników, ofertami oraz przeglądanie rezerwacji (rys. \ref{fig:adminpanel}).

\begin{figure}[H]
    \centering
    \includegraphics[width=0.9\textwidth]{img/admin_panel.png}
    \caption{Panel administratora z listą użytkowników}
    \label{fig:adminpanel}
\end{figure}

\section{Walidacja danych}

System przeprowadza walidację danych wejściowych zarówno po stronie klienta (Java Swing), jak i po stronie bazy danych. Przykładowe mechanizmy walidacyjne obejmują:

\begin{itemize}
    \item sprawdzenie poprawności adresu e-mail,
    \item weryfikację długości oraz zgodności hasła,
    \item unikalność loginu użytkownika (na poziomie bazy),
    \item poprawność danych liczbowych (np. zakres cenowy oferty).
\end{itemize}

\chapter{Harmonogram i zarządzanie projektem}
\label{cha:Harmonogram i zarządzanie projektem}

Proces realizacji projektu został rozplanowany w sposób etapowy, z uwzględnieniem projektowania bazy danych, implementacji funkcjonalności aplikacji, testowania oraz przygotowania dokumentacji. W niniejszym rozdziale przedstawiono harmonogram prac, opis wykorzystanego systemu kontroli wersji oraz graficzną reprezentację przebiegu projektu.

Podział zadań i organizacja pracy miały na celu sprawną realizację założeń projektowych przy zachowaniu przejrzystości etapów i kontroli postępów. Poszczególne zadania były wykonywane w sposób iteracyjny, przy czym szczególny nacisk położono na testowanie aplikacji oraz zachowanie jej spójności z dokumentacją.

% ------------------------------------------------------------------------

\section{Etapy realizacji}

Proces tworzenia aplikacji został podzielony na trzy główne etapy, obejmujące kolejno: projektowanie, implementację oraz testowanie i dokumentację. Harmonogram został dostosowany do założeń semestralnych, a prace były prowadzone iteracyjnie z możliwością nanoszenia poprawek na bieżąco.

\begin{itemize}
    \item \textbf{Etap 1 – Projektowanie}  
    Obejmował analizę wymagań funkcjonalnych, opracowanie struktury bazy danych, określenie architektury aplikacji oraz zaprojektowanie głównych komponentów interfejsu użytkownika. Ten etap zrealizowano w pierwszej połowie maja 2025 roku.

    \item \textbf{Etap 2 – Implementacja}  
    Polegał na tworzeniu kodu aplikacji w języku Java, budowie GUI z użyciem biblioteki Swing oraz integracji z bazą danych PostgreSQL przy wykorzystaniu JDBC. Prace koncentrowały się także na podziale funkcji pomiędzy użytkownika a administratora. Etap realizowano od połowy maja do początku czerwca 2025 roku.

    \item \textbf{Etap 3 – Testowanie i dokumentacja}  
    Ostatnia faza obejmowała testy ręczne oraz przygotowanie wybranych testów jednostkowych dla krytycznych klas. Równolegle tworzono dokumentację techniczną zgodną z przyjętym szablonem uczelnianym. Etap zakończył się w drugiej połowie czerwca 2025 roku.
\end{itemize}

% ------------------------------------------------------------------------

\section{Diagram Gantta}

W celu zobrazowania przebiegu realizacji poszczególnych etapów projektu przygotowano diagram Gantta, który przedstawia planowany oraz rzeczywisty harmonogram działań w ramach projektu. Diagram umożliwia szybki podgląd czasu trwania każdego z etapów oraz ich wzajemnych zależności czasowych.

Na Rys. \ref{fig:gantt_diagram} przedstawiono przebieg prac nad projektem aplikacji biura podróży w układzie tygodniowym.

\begin{figure}[H]
    \centering
    \includegraphics[width=0.85\textwidth]{img/gantt.png}
    \caption{Diagram Gantta przedstawiający harmonogram realizacji projektu}
    \label{fig:gantt_diagram}
\end{figure}

% ------------------------------------------------------------------------

\section{Repozytorium GitHub i wersjonowanie}

Podczas realizacji projektu wykorzystano system kontroli wersji \texttt{Git}, który umożliwia śledzenie zmian w kodzie źródłowym, cofanie się do wcześniejszych wersji oraz pracę w gałęziach rozwojowych bez ryzyka naruszenia głównej wersji aplikacji. Przechowywanie historii wersji pozwoliło również na lepszą organizację pracy oraz bieżące wprowadzanie poprawek bez utraty stabilnych funkcjonalności.

Kod źródłowy aplikacji umieszczono w publicznym repozytorium platformy GitHub. Główna gałąź \texttt{main} zawiera stabilną wersję aplikacji, natomiast w trakcie prac wykorzystywano również tymczasowe gałęzie robocze, m.in. \texttt{gui}, \texttt{dao\_integration} oraz \texttt{fixes}, które służyły do rozwijania poszczególnych modułów.

Adres repozytorium projektu:
\begin{center}
\url{https://github.com/Arioch2/Biuro_Podr-y}
\end{center}

Zastosowanie systemu Git zapewniło pełną kontrolę nad projektem i umożliwiło jego bezpieczne rozwijanie w trybie iteracyjnym.

\chapter{Harmonogram, testy i podsumowanie}

\section{Harmonogram realizacji projektu}

Projekt został zrealizowany w okresie od maja do czerwca 2025 roku. Całość prac została podzielona na trzy główne etapy:

\begin{itemize}
    \item \textbf{maj 2025} — zaprojektowanie struktury bazy danych oraz graficznego interfejsu użytkownika,
    \item \textbf{maj–czerwiec 2025} — implementacja logiki aplikacji oraz integracja z bazą danych PostgreSQL,
    \item \textbf{czerwiec 2025} — przeprowadzenie testów, wprowadzenie poprawek oraz przygotowanie dokumentacji technicznej.
\end{itemize}

Wizualny przebieg realizacji przedstawiono na diagramie Gantta (rys.~\ref{fig:gantt}).

\begin{figure}[H]
    \centering
    \includegraphics[width=0.95\textwidth]{img/gantt.png}
    \caption{Diagram Gantta przedstawiający harmonogram projektu}
    \label{fig:gantt}
\end{figure}

\section{Repozytorium i wersjonowanie}

Do kontroli wersji zastosowano system \texttt{Git}, natomiast kod źródłowy przechowywano w publicznym repozytorium serwisu GitHub:

\begin{itemize}
    \item \textbf{Repozytorium:} \url{https://github.com/Arioch2/Biuro_Podr-y}
    \item \textbf{Główna gałąź:} \texttt{main}
\end{itemize}

Rozwiązanie to umożliwia łatwą kontrolę nad postępem prac oraz historią zmian.

\section{Testowanie aplikacji}

Testy zostały przeprowadzone zarówno manualnie, jak i poprzez proste testy jednostkowe dla wybranych fragmentów logiki biznesowej. Sprawdzono poprawność działania formularzy, walidacji danych oraz połączenia z bazą danych.

\subsection{Walidacja danych wejściowych}

Walidacja danych odbywa się głównie po stronie klienta z wykorzystaniem klasy \texttt{Validator.java}. Przykładowy fragment tej klasy przedstawiono na listingu \ref{lst:validator}.

\lstinputlisting[style=javaStyle, caption={Fragment klasy \texttt{Validator}}, label={lst:validator}]{src/Validator.java}

\subsection{Połączenie z bazą danych}

Kluczową rolę w obsłudze połączenia z bazą danych pełni klasa \texttt{DatabaseConnector}. Listing \ref{lst:dbconnector} prezentuje fragment tej klasy, odpowiedzialny za inicjalizację połączenia JDBC.

\lstinputlisting[style=javaStyle, caption={Fragment klasy \texttt{DatabaseConnector}}, label={lst:dbconnector}]{src/DatabaseConnector.java}

\section{Podsumowanie i wnioski}

Zaprojektowana i zaimplementowana aplikacja spełnia wszystkie zakładane cele funkcjonalne. Użytkownik końcowy ma możliwość rejestracji, logowania, przeglądania i rezerwacji ofert. Z kolei administrator systemu otrzymuje dostęp do rozszerzonych funkcji zarządzania.

Potencjalne kierunki dalszego rozwoju systemu obejmują:

\begin{itemize}
    \item implementację mechanizmu wysyłania potwierdzeń e-mailowych po dokonaniu rezerwacji,
    \item rozszerzenie systemu o możliwość eksportowania danych do plików PDF,
    \item integrację z zewnętrznymi usługami, takimi jak mapy lub prognozy pogody,
    \item dostosowanie interfejsu użytkownika do urządzeń mobilnych.
\end{itemize}

\section{Pozostałe widoki aplikacji}

Poza głównym panelem użytkownika i administracyjnym menu systemu, aplikacja udostępnia szereg dodatkowych interfejsów, wspierających funkcjonalność związaną z kontem użytkownika oraz zarządzaniem systemem.

Na rysunku \ref{fig:changepass} przedstawiono panel umożliwiający zmianę hasła użytkownika.

\begin{figure}[H]
    \centering
    \includegraphics[width=0.85\textwidth]{img/ChangePasswordPanel.png}
    \caption{Panel zmiany hasła użytkownika}
    \label{fig:changepass}
\end{figure}

Dane użytkownika mogą być edytowane za pośrednictwem dedykowanego formularza (rys.~\ref{fig:edituser}), a konto może zostać usunięte w panelu z rysunku \ref{fig:deleteuser}.

\begin{figure}[H]
    \centering
    \includegraphics[width=0.85\textwidth]{img/UpdateUserPanel.png}
    \caption{Panel edycji danych użytkownika}
    \label{fig:edituser}
\end{figure}

\begin{figure}[H]
    \centering
    \includegraphics[width=0.85\textwidth]{img/DeleteAccountPanel.png}
    \caption{Panel usuwania konta}
    \label{fig:deleteuser}
\end{figure}

Lista dokonanych rezerwacji dostępna jest z poziomu panelu użytkownika (rys.~\ref{fig:userreservations}).

\begin{figure}[H]
    \centering
    \includegraphics[width=0.85\textwidth]{img/ReservationsPanel.png}
    \caption{Panel rezerwacji użytkownika}
    \label{fig:userreservations}
\end{figure}

Administrator posiada również dostęp do rozszerzonych interfejsów zarządzania systemem. Na rysunkach \ref{fig:adminoffers}–\ref{fig:adminstats} zaprezentowano odpowiednio panele zarządzania ofertami, użytkownikami, rezerwacjami oraz statystykami.

\begin{figure}[H]
    \centering
    \includegraphics[width=0.85\textwidth]{img/OffersManagementPanel.png}
    \caption{Panel zarządzania ofertami (administrator)}
    \label{fig:adminoffers}
\end{figure}

\begin{figure}[H]
    \centering
    \includegraphics[width=0.85\textwidth]{img/UserManagmentPanel.png}
    \caption{Panel zarządzania użytkownikami (administrator)}
    \label{fig:adminusers}
\end{figure}

\begin{figure}[H]
    \centering
    \includegraphics[width=0.85\textwidth]{img/ReservationsManagementPanel.png}
    \caption{Panel zarządzania rezerwacjami (administrator)}
    \label{fig:adminreservations}
\end{figure}

\begin{figure}[H]
    \centering
    \includegraphics[width=0.85\textwidth]{img/StatisticPanel.png}
    \caption{Panel statystyk systemowych — liczba rezerwacji i aktywnych użytkowników}
    \label{fig:adminstats}
\end{figure}



% Wyłączenie działania `ulem` na czas bibliografii
\renewcommand{\emph}[1]{\textit{#1}}
% Bibliografia
% Dodanie bibliografi do spisu treści
\addcontentsline{toc}{section}{\textbf{Bibliografia}}
\bibliographystyle{plain}
\bibliography{bibliografia}

% Przywrócenie działania `ulem`
\renewcommand{\emph}[1]{\uline{#1}}

\clearpage
% Dodanie spisu rysunków do spisu treści
\addcontentsline{toc}{section}{\textbf{Spis rysunków}}
\listoffigures
\clearpage

% Dodanie spisu tabel do spisu treści
\addcontentsline{toc}{section}{\textbf{Spis tabel}}
\listoftables
\clearpage


\clearpage

% Dodanie spisu listingow do spisu treści
\addcontentsline{toc}{section}{\textbf{Spis listingów}}
\lstlistoflistings
\clearpage


% \appendix
\chapter*{}
\label{cha:statement-A}
\makeatletter
\addcontentsline{toc}{section}{\textbf{Oświadczenie studenta o samodzielności pracy}}

\noindent
\begin{flushright}
    \begin{minipage}[!h]{10cm}
        Załącznik nr 2 do Zarządzenia nr 228/2021 Rektora Uniwersytetu Rzeszowskiego z dnia 1 grudnia 2021 roku w sprawie ustalenia procedury antyplagiatowej w Uniwersytecie Rzeszowskim
    \end{minipage}
\end{flushright}

\begin{center}
    \vspace*{10mm}
    \noindent  {\textbf{OŚWIADCZENIE STUDENTA O SAMODZIELNOŚCI PRACY} }
    \vspace*{10mm}
\end{center}

\noindent
\dotuline{\hspace{1.3cm}\@author\hspace{1.3cm}}\\ % Linia pozioma
{\small Imię (imiona) i nazwisko studenta }\\

\noindent \@faculty\\

\noindent \dotuline{\hspace{1.4cm}\@degreeprogramme \hspace{1.4cm}}\\
{\small Nazwa kierunku} \\

\noindent \dotuline{\hspace{1.8cm}\@noAlbum\hspace{1.9cm}}\\
{\small Numer albumu}

\begin{enumerate}
    \item Oświadczam, że moja praca projektowa pt.: \@titlePL
          \begin{enumerate}[label=\arabic*)
              \item została przygotowana przeze mnie samodzielnie*,
              \item nie narusza praw autorskich w rozumieniu ustawy z dnia 4 lutego 1994 roku o prawie autorskim i prawach pokrewnych (t.j. Dz.U. z 2021 r., poz. 1062) oraz dóbr osobistych chronionych prawem cywilnym,
              \item nie zawiera danych i informacji, które uzyskałem/am w sposób niedozwolony,
              \item nie była podstawą otrzymania oceny z innego przedmiotu na uczelni wyższej ani mnie, ani innej osobie.
          \end{enumerate}
    \item Jednocześnie wyrażam zgodę/nie wyrażam zgody** na udostępnienie mojej pracy projektowej do celów naukowo--badawczych z poszanowaniem przepisów ustawy o prawie autorskim i prawach pokrewnych.
\end{enumerate}


\vspace*{10mm}

\noindent
\underline{\hspace{6cm}} \hfill \underline{\hspace{6cm}} \\ % Puste miejsce na miejscowość, data oraz podpis
\hspace*{13mm}(miejscowość, data)  \hspace*{63mm}(czytelny podpis studenta)
\vspace*{10mm}

\vfill
\noindent
* Uwzględniając merytoryczny wkład prowadzącego przedmiot \\
** -- niepotrzebne skreślić



\end{document}
